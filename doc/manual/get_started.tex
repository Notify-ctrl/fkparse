\chapter{开始使用}

解压缩下载好了的fkparse后,你应该会得到这些文件:

\begin{itemize}
 \item \textbf{manual.pdf}: 本手册
 \item \textbf{fkparse\_qt.exe}: 主程序
 \item \textbf{example/}: 包含示例文本和示例素材的文件夹
 \item \textbf{LICENSE}: GPLv3许可证文件
\end{itemize}

除此之外,你还要确保自己拥有太阳神三国杀游戏主程序。

\section{编译运行示例文件}

首先打开fkparse\_qt.exe,看到如下界面(补个插图)

然后先点击Choose File...,选择example文件夹下面的example.txt,然后点击Compile按钮进行编译。如果编译出错,错误将出现在下方,否则下面会告诉你编译成功了。

编译成功后点击一下Pack按钮,程序就会自动整合所有素材,然后在输入的txt的所在文件夹下面生成一个新的文件夹。将新文件夹下面的四个文件夹直接复制到太阳神三国杀的根目录下面,然后启动神杀即可。

至此,你已经知道如何使用fkparse了,接下来说明example.txt这种文件所要求的语法格式。

\section{编写自己的第一个拓展文件}

首先,新建一个文本文件,就起名为study.txt吧。在之后的部分都将基于这个文件进行操作。打开study.txt,用记事本或者代码编辑器之类的东西都可。

\subsection{创建拓展包}

拓展文件首先要从创建拓展包开始。fkparse中创建拓展包的格式为:\\

\emph{拓展包 <标识符>} \\

“标识符”就是用单引号括起来的文本,毕竟中文不用空格进行分词。一个拓展文件中可以含有多个拓展包,下面我们来创建两个空拓展包:

\begin{verbatim}
 拓展包 '学习包1'
 拓展包 '学习包2'
\end{verbatim}

保存,然后用它去生成study.lua。将study.lua放入extensions里面,你应该能在游戏中看到这两个拓展包了。

\subsection{创建武将}

创建武将的详细的内容在下一章说明。这里我们只是简单的新建一个武将而已。

现在我们要创建一个武将,它所属于学习包1,名字是猪八戒,称号是净坛使者,神势力,24体力。那么我们现在在“拓展包 '学习包1'”下面另起一个新行,然后输入:

\begin{verbatim}
 # 神 "净坛使者" '猪八戒' 24 []
\end{verbatim}

这一段话的详细内容下一章再细说。现在保存文件,然后重新生成lua。现在你的study.txt的内容应该是像这样:

\begin{verbatim}
拓展包 '学习包1'
# 神 "净坛使者" '猪八戒' 24 []
拓展包 '学习包2'
\end{verbatim}

\section{错误处理}

如果编译出错,错误信息将显示在窗口内部。请根据报错信息仔细检查出错或者咨询作者吧。
