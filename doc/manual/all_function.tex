\chapter{所有的预定义函数}

fkparse提供了一些内置函数供您使用。

\section{随机数}

函数名称:\verb|'生成随机数'|

功能:生成一个范围为[下界,上界]的随机数。

参数:

\begin{itemize}
  \item \verb|'上界'|:数字类型,默认为10
  \item \verb|'下界'|:数字类型,默认为1
\end{itemize}

返回:数字类型

\section{提示信息}

函数名称:\verb|'创建提示信息'|

功能:根据给定的各种参数,构造一个提示信息。

参数:

\begin{itemize}
  \item \verb|'文本'|:字符串类型,无默认值
  \item \verb|'玩家1'|:玩家类型,默认为空
  \item \verb|'玩家2'|:玩家类型,默认为空
  \item \verb|'变量1'|:数字类型或者字符串类型,默认为空
  \item \verb|'变量2'|:数字类型或者字符串类型,默认为空
\end{itemize}

返回:字符串类型

说明:\\

在这个函数中,提示信息的主体由\verb|'文本'|决定。在\verb|'文本'|中,可以通过如下方式向文本中引入变量:

\begin{itemize}
  \item \verb|%src|: 对应着参数\verb|'玩家1'|,它会被替换为对应玩家的武将名称
  \item \verb|%dest|: 对应着参数\verb|'玩家2'|,它会被替换为对应玩家的武将名称
  \item \verb|%arg|: 对应着参数\verb|'变量1'|,被替换为对应的值
  \item \verb|%arg2|: 对应着参数\verb|'变量2'|,被替换为对应的值
\end{itemize}

\section{卡牌规则}

函数名称:\verb|'创建卡牌规则'|

功能:根据给定的各种参数,构造一个能判定卡牌是否符合类型的字符串。

参数:

\begin{itemize}
  \item \verb|'牌名表'|:保存着所有可行牌名的列表,默认为所有牌名
  \item \verb|'花色表'|:保存着所有可行花色的列表,默认为所有牌名
  \item \verb|'点数表'|:保存着所有可行点数的列表,默认为所有牌名
\end{itemize}

返回:字符串类型

\section{虚拟牌}

函数名称:\verb|'创建虚拟牌'|

功能:根据给定的各种参数,构造一张虚拟牌。注意这边“虚拟牌”和规则集说的“虚拟牌”不是一个东西。

参数:

\begin{itemize}
  \item \verb|'点数'|:数字类型,保存着虚拟牌的点数,默认由游戏自行判断,一般可不填
  \item \verb|'花色'|:字符串类型,虚拟牌的花色,默认由游戏自行判断,一般可不填
  \item \verb|'牌名'|:字符串类型,即虚拟牌的牌名,默认为普通杀
  \item \verb|'子卡牌'|:卡牌数组类型,保存着虚拟牌的子卡牌(即其对应的实体卡),默认为空
  \item \verb|'技能名'|:字符串类型,表示创建该虚拟牌的技能名,默认为空字符串
\end{itemize}

说明:\\

太阳神三国杀“自行判断”花色和点数的行为可能与规则集有所出入,特于此说明。

\begin{itemize}
 \item 虚拟牌的颜色取决于所有子卡牌的颜色,只要子卡牌的颜色都相同,那么虚拟牌也是那个颜色。否则虚拟牌的颜色为无色。除非子卡牌刚好只有一张,虚拟牌都是没有花色的。
 \item 虚拟牌的点数为所有子卡牌点数之和(最大为K),若没有子卡牌则没有点数。
\end{itemize}

返回:卡牌类型
