\chapter{表达式和语句}

要讨论语句,自然避免不了表达式了。我们先来看看表达式。

\section{表达式}

表达式用来产生一个值。它可能本身就是一个值,也可能是值与值之间进行加减之类的运算,也可能是执行一个动作获得返回值之类的。

\subsection{数值型表达式}

数值型表达式,说白了就是一个值。它具体又分为这几种类型:

\subsubsection{布尔型}

布尔型就是\emph{真、假}这两个字。实践表明这两个字几乎不会被用到。

\subsubsection{数量型}

一个整数,比如1376之类的。很遗憾只能是整数,但三国杀里面也基本用不着小数来着。

\subsubsection{字符串型}

一个字符串。

\subsubsection{变量}

一个变量。关于变量,将在下一节进行讨论。

\subsubsection{数组}

跟C语言的数组的格式差不多,fkparse中数组的格式为

\begin{verbatim}
 '[' [<表达式> {,<表达式>}] ']'
\end{verbatim}

一对\textbf{半角}方括号,中间是零个或者多个表达式。如果是多个的话,中间要有逗号隔开。

这里说明一下,fkparse的数组里面,所有成员的类型必须相同,否则就报错。

\subsection{运算型表达式}

表达式之间支持各种运算。

\subsubsection{四则运算}

加减乘除四种运算,对应的符号分别是\emph{+、-、*、/}。比如2+3,5*3等等。

这个运算要求两个参与运算的表达式都必须是数字类型。运算结果也为数字类型。

\subsubsection{比较运算}

可以对两个数字类型的表达式作比较运算,有以下几种运算符。

\begin{itemize}
 \item \emph{大于}:判断前一个是否大于后一个
 \item \emph{小于}:判断前一个是否小于后一个
 \item \emph{不是}:判断两个表达式是否不相等。这个运算符对两侧的表达式没有类型要求
 \item \emph{是}:判断两个表达式是否相等。这个运算符对两侧的表达式没有类型要求
 \item \emph{不小于}:判断前一个是否不小于后一个
 \item \emph{不大于}:判断前一个是否不大于后一个
\end{itemize}

运算的结果为布尔类型。运算符直接就是汉字,使用例:2大于3。

\subsubsection{逻辑运算}

两个布尔类型的表达式可以进行逻辑运算。运算符如下:

\begin{itemize}
 \item \emph{且}:若两个都是真,则结果为真,否则结果为假
 \item \emph{或}:只要其中有一个是真,则结果为真;若两者都为假,则结果为假
\end{itemize}

运算的结果为布尔类型。

\subsection{函数调用}

详见函数调用那一章。

\subsection{动作语句表达式}

光有上面几种还无法取得所有想要的值,有时候还需要借助动作语句的力量。

动作语句表达式的格式为:

\begin{verbatim}
 '(' <动作语句> ')'
\end{verbatim}

也就是用一对\textbf{半角}圆括号括起来的动作语句。有关动作语句的详情,稍后会进行描述。\\

接下来介绍变量。

\section{变量}

变量有三种形式:

\begin{enumerate}
 \item \verb|<标识符>|。一般是先通过赋值语句定义,然后再进行使用。
 \item \verb|<标识符> 的 <字符串>|。用来获得<标识符>对应的那个变量的某一个属性,比如获得卡牌的花色点数之类的。有关于属性列表的信息,可以在参考手册部分找到,这里不罗列。
 \item \verb|'(' <表达式> ')' 的 <字符串>|。有时候需要获得属性的是一个表达式,此时需要把表达式用括号括起来。
\end{enumerate}

所有的变量都需要定义后使用,但预定义好了的除外。预定义变量又分为两种:

\begin{enumerate}
 \item 全局有效的预定义变量,不能重新赋值
 \item 根据触发时机而确定的跟相关时机有关的变量,可以重新赋值。(但只有某些时机时候重新赋值才有效)比如\verb|'伤害来源'|。
\end{enumerate}

\section{语句}

表达式用来产生值,而语句则是具体执行的操作。

\subsection{赋值语句}

赋值语句的格式为:

\begin{verbatim}
 令 <变量> 为 <表达式>
\end{verbatim}

这个语句通常用来定义变量,也可以改变变量的值。比如\verb|令'Y'为3|就是一个合法的赋值语句。

注意,你不能对系统内置的变量进行重新赋值,否则会报错。如果赋值的变量是形如“xx的xx”这种格式的话,fkparse可能检测不出错误,但这种行为生成的代码可能会无法正常执行。

\subsection{判断语句}

if语句在几乎任何语言里面都有,fkparse也不例外,判断语句的格式为:

\begin{verbatim}
 若 <表达式> 则 <语句块> { 否则若 <表达式> 则 <语句块> } [否则 <语句块>] 以上
\end{verbatim}

\subsection{循环语句}

循环语句的格式如下:

\begin{verbatim}
 重复此流程: <语句块> 直到 <表达式>
\end{verbatim}

注意,“重复此流程:”中的冒号是\textbf{半角}的。为什么只提供这一种循环形式呢,这是因为三国杀的技能描述里面基本都是把某个行为做一遍,然后再重复此流程直到什么什么之类的,比如洛神、OL吉占之类的技能。

\subsection{break语句}

break语句用来中止一个循环,其格式为:

\begin{verbatim}
 中止此流程
\end{verbatim}

请确保自己只在循环语句的内部使用这个,fkparse不会去检查的。如果你在别的地方使用的话,生成的lua会导致游戏无法启动哦。

\subsection{返回语句}

返回语句用来让函数带上一个返回值,格式为:

\begin{verbatim}
返回 <表达式>
\end{verbatim}

这个语句只在定义函数的时候用得到,关于函数定义在后续章节中有讲到。

\subsection{函数调用}

详见函数那一章。

\subsection{数组操作}

详见后文。

\subsection{动作语句}

终于到所有语句中的重头戏——动作语句了。动作语句用来执行一个实际的动作,比如摸牌、打伤害之类的,有些动作语句还能产生值。

但是由于动作语句的类型太多了,在这里罗列不太现实。请各位去查看参考手册吧。

