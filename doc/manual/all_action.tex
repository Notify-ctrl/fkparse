\chapter{所有的动作语句}

先列出语句格式,再依次说明表达式类型需求,最后说返回值(如果有)。

前面有个没说过的,动作语句也可以带额外参数,毕竟一句话没法描述完全部的情况。因此对于一部分动作语句也提供了一些额外的参数,像函数那样用就行了,具体参考已有的txt。

\section{摸牌}

\begin{verbatim}
<表达式> 摸 <表达式> 张牌
\end{verbatim}

玩家类型;数字类型

\section{失去体力}

\begin{verbatim}
<表达式> 失去 <表达式> 点体力
\end{verbatim}

玩家类型;数字类型

\section{失去体力上限}

\begin{verbatim}
<表达式> 失去 <表达式> 点体力上限
\end{verbatim}

玩家类型;数字类型

\section{造成伤害}

\begin{verbatim}
<表达式> 对 <表达式> 造成 <表达式> 点伤害
\end{verbatim}

玩家类型;玩家类型;数字类型 \\

额外参数:

\begin{itemize}
  \item \verb|'伤害属性'|:数字类型,默认为\verb|'无属性'|
  \item \verb|'造成伤害的牌'|:卡牌类型,默认为\verb|nil|
  \item \verb|'造成伤害的原因'|:字符串类型,默认为空字符串
\end{itemize}

\section{受到伤害}

\begin{verbatim}
<表达式> 受到 <表达式> 点伤害
\end{verbatim}

玩家类型;数字类型 \\

额外参数同上。

\section{回复体力}

\begin{verbatim}
<表达式> 回复 <表达式> 点体力
\end{verbatim}

玩家类型;数字类型 \\

额外参数:

\begin{itemize}
  \item \verb|'回复来源'|:玩家类型,默认为\verb|nil|
  \item \verb|'回复的牌'|:卡牌类型,默认为\verb|nil|
\end{itemize}

\section{回复体力上限}

\begin{verbatim}
<表达式> 回复 <表达式> 点体力上限
\end{verbatim}

玩家类型;数字类型

\section{获得技能}

\begin{verbatim}
<表达式> 获得技能 <表达式>
\end{verbatim}

玩家类型;字符串类型

\section{失去技能}

\begin{verbatim}
<表达式> 失去技能 <表达式>
\end{verbatim}

玩家类型;字符串类型

\section{获得标记}

\begin{verbatim}
<表达式> 获得 <表达式> 枚 <字符串> [隐藏] 标记
\end{verbatim}

玩家类型;数字类型

<字符串>表示标记的名称。

注:“隐藏”可填可不填,如果写了是隐藏标记的话,获得标记将不会显示在战报中,也没有图片,可以借此实现技能发动次数控制等。

\section{失去标记}

\begin{verbatim}
<表达式> 失去 <表达式> 枚 <字符串> [隐藏] 标记
\end{verbatim}

玩家类型;数字类型

\section{统计标记数量}

\begin{verbatim}
<表达式> <字符串> [隐藏] 标记数量
\end{verbatim}

玩家类型

返回:数字类型

\section{询问选择选项}

\begin{verbatim}
<表达式> 从 <表达式> 选择一项
\end{verbatim}

玩家类型;数组(字符串类型) \\

额外参数:

\begin{itemize}
  \item \verb|'选择的原因'|:字符串类型,默认为空字符串
\end{itemize}

返回:字符串类型

\section{询问选择玩家}

\begin{verbatim}
<表达式> 从 <表达式> 选择一名角色
\end{verbatim}

玩家类型;数组(玩家类型) \\

额外参数:

\begin{itemize}
  \item \verb|'选择的原因'|:字符串类型,默认为空字符串
  \item \verb|'提示框文本'|:字符串类型,默认为默认的提示文本
  \item \verb|'可以点取消'|:布尔类型,默认为\verb|真|
  \item \verb|'提示技能发动'|:布尔类型,默认为\verb|假|
\end{itemize}

返回:玩家类型

\section{询问发动技能}

\begin{verbatim}
<表达式> 选择发动 <字符串>
\end{verbatim}

玩家类型

<字符串>是技能的中文名字,且只能是本文件中已经定义的技能。定义的先后顺序不重要

返回:布尔类型

\section{获得卡牌}

\begin{verbatim}
<表达式> 获得卡牌 <表达式>
\end{verbatim}

玩家类型;卡牌类型 \\

额外参数:

\begin{itemize}
  \item \verb|'公开'|:布尔类型,默认为\verb|真|
\end{itemize}

\section{拥有技能}

\begin{verbatim}
<表达式> 拥有技能 <字符串>
\end{verbatim}

玩家类型

<字符串>是技能的中文名字,且只能是本文件中已经定义的技能。定义的先后顺序不重要

返回:布尔类型

\section{因发动技能而弃牌}

\begin{verbatim}
<表达式> 因技能 <字符串> 弃置卡牌 <表达式>
\end{verbatim}

玩家类型;卡牌数组

<字符串>是技能的中文名字,且只能是本文件中已经定义的技能。定义的先后顺序不重要

本语句只能用在主动技的效果中。

\section{主动技的发动次数}

\begin{verbatim}
<表达式> 发动主动技 <字符串> 的次数
\end{verbatim}

玩家类型;卡牌数组

<字符串>是技能的中文名字,且只能是本文件中已经定义的技能。定义的先后顺序不重要

这种办法只能获取当前阶段里面发动那个技能的次数。如果想要做一回合发动多少次的技能,请使用隐藏标记实现。

\section{令角色弃牌}

\begin{verbatim}
<表达式> 弃置 <表达式> 张牌
\end{verbatim}

玩家类型;数字类型 \\

额外参数:

\begin{itemize}
  \item \verb|'技能名'|: 发起这次弃牌的技能名,默认为空字符串。
  \item \verb|'最小弃置数量'|: 数字类型,默认为要求弃牌数量的值。
  \item \verb|'可以点取消'|: 布尔类型,是否可以点击取消拒绝弃牌,默认为\verb|假|。
  \item \verb|'可以弃装备'|: 布尔类型,是否可以弃置装备牌,默认为\verb|真|。
  \item \verb|'提示信息'|: 字符串类型,默认为空字符串(默认的提示信息)。
  \item \verb|'弃牌规则'|: 字符串类型,默认为无限制。
\end{itemize}

返回类型:卡牌数组,即目标角色弃置了的牌,可能是空的数组。

\section{播放台词}

\begin{verbatim}
  <表达式> 说出 <字符串> 的台词
\end{verbatim}

如果有多个编号完的音频还需要选择,在后面加上\{'音频编号': <编号>\}。

\section{交换座位}

\begin{verbatim}
  <表达式> 与 <表达式> 交换座位
\end{verbatim}

表达式均为玩家类型。

没有其他参数。

\section{洗牌}

\begin{verbatim}
  <表达式> 洗牌
\end{verbatim}

表达式均为玩家类型。

没有其他参数。

\section{变身}

\begin{verbatim}
  <表达式> 变身为 <字符串>
\end{verbatim}

表达式为玩家类型,字符串为要变身的武将,应使用内部标识。

其他参数:

\begin{itemize}
  \item \verb|'是否满状态'| 布尔类型,默认为\verb|真|
  \item \verb|'是否以开始游戏状态变身'| 布尔类型,默认为\verb|真|
  \item \verb|'是否是变更副将'| 布尔类型,默认为\verb|假|(即变更主将,而且一般也只能变更主将)
  \item \verb|'是否发送信息'| 布尔类型,默认为\verb|真|
\end{itemize}

\section{判定}

\begin{verbatim}
  <表达式> 判定
\end{verbatim}

玩家类型 \\

额外参数:

\begin{itemize}
  \item \verb|'技能名'|: 发起这次判定的技能名,默认为空字符串。
  \item \verb|'最小弃置数量'|: 数字类型,默认为要求弃牌数量的值。
  \item \verb|'判定规则'|: 字符串类型,表示判定牌需要的某种规则,默认为任意卡牌。
  \item \verb|'希望判定中'|: 布尔类型,是否希望获得\verb|'判定规则'|中描述的判定结果,默认为\verb|真|。
  \item \verb|'提示信息'|: 布尔类型,是否会播放打钩打叉动画,默认为\verb|真|。
\end{itemize}

\section{类观星技能}

适用于“将一些牌以任意顺序放在牌堆顶/牌堆底/两者都有”的场合。

\begin{verbatim}
  <表达式> 对 <表达式> 进行观星
\end{verbatim}

第一个表达式为执行观星的玩家,第二个表达式为此技能处理的牌列表。

额外参数:

\begin{itemize}
  \item \verb|'观星类型'|: 指定卡组能够放置的位置,共三种选择:\verb|'顶部底部均放置'|、\verb|'只放置顶部'|、\verb|'只放置底部'|。默认为\verb|'顶部底部均放置'|。
\end{itemize}

\section{选取牌堆顶X张牌}

\begin{verbatim}
  <表达式> 选择牌堆顶 <表达式> 张牌
\end{verbatim}

第一个表达式为执行行动的玩家,第二个表达式为牌的数量。

额外参数:

\begin{itemize}
  \item \verb|'是否不放回'|: 布尔类型,被选取的卡牌是否在处理结束后直接置入弃牌堆。默认为\verb|'真'|。
\end{itemize}

\section{改判}

\begin{verbatim}
  <表达式> 将判定结果修改为 <表达式>
\end{verbatim}

第一个是玩家,第二个是要修改的牌。

额外参数:

\begin{itemize}
  \item \verb|'技能名'|: 字符串类型,处理区改判牌显示的技能名。默认没有,但是建议写成技能的名字。
  \item \verb|'是否交换'|: 布尔类型,是否将原来的判定牌与改判的牌置换。默认为\verb|'假'|。
\end{itemize}

特别注意:调用此语句需将时机设定为“改判前”。

\section{要求选择自己一张牌}

\begin{verbatim}
  <表达式> 选择自己的一张牌
\end{verbatim}

玩家类型

令玩家选择自己的任意一张牌。

额外参数:

\begin{itemize}
  \item \verb|'选牌规则'| 字符串类型,表示选牌的具体规则,默认为任意卡牌。
  \item \verb|'提示'| 字符串类型,为选牌的提示信息,默认为默认的提示。
  \item \verb|'技能名'| 字符串类型,默认为当前的技能。
\end{itemize}

\section{要求使用一张牌}

\begin{verbatim}
  <表达式> 使用一张牌
\end{verbatim}

玩家类型。

额外参数:

\begin{itemize}
  \item \verb|'选牌规则'| 字符串类型,表示选牌的具体规则,默认为任意卡牌。
  \item \verb|'提示'| 字符串类型,为选牌的提示信息,默认为空。
  \item \verb|'目标'| 玩家类型,本次要求使用卡牌需指定的目标。
  \item \verb|'技能名'| 字符串类型,默认为空。
\end{itemize}

\section{要求打出一张牌}

\begin{verbatim}
  <表达式> 打出一张牌
\end{verbatim}

玩家类型。

额外参数:

\begin{itemize}
  \item \verb|'选牌规则'| 字符串类型,表示选牌的具体规则,默认为任意卡牌。
  \item \verb|'提示'| 字符串类型,为选牌的提示信息,默认为空。
  \item \verb|'是否为改判'| 布尔类型,默认为假。
  \item \verb|'技能名'| 字符串类型,默认为空。
\end{itemize}

\section{选择他人一张牌}

\begin{verbatim}
  <表达式> 选择 <表达式> 一张牌
\end{verbatim}

两个参数均为玩家类型

用来令玩家1选择玩家2的一张牌,像过河拆桥的弹窗那样。

额外参数:

\begin{itemize}
  \item \verb|'位置'| 数字数组,表示可以被选牌的区域,默认为只有手牌区。
  \item \verb|'原因'| 字符串类型,表示被选牌的原因,默认为空。
  \item \verb|'是否可见手牌'| 布尔类型,默认为假。
\end{itemize}

\section{聊天}

\begin{verbatim}
  <表达式> 说出 <表达式>
\end{verbatim}

玩家类型;字符串类型/数字类型

本语句可以让一名玩家在聊天框中发送一句话。

\section{发送战报}

\begin{verbatim}
  <表达式> 发送战报 <表达式>
\end{verbatim}

玩家类型;字符串类型 \\

额外参数:

\begin{itemize}
  \item \verb|'%from'|: 玩家类型。将以玩家使用的武将名替换战报文本的所有“\%from”。
  \item \verb|'%to'|: 玩家类型。将以玩家使用的武将名替换战报文本的所有“\%to”。
  \item \verb|'%card'|: 卡牌类型。将以形如“杀[♣7]”的形式替换战报文本中的所有“\%card”。
  \item \verb|'%arg'|: 任意类型。将替换战报文本中的所有“\%arg”。
  \item \verb|'%arg2'|: 任意类型。将替换战报文本中的所有“\%arg2”。
\end{itemize}

\section{弃置牌}

\begin{verbatim}
  <表达式> 弃置牌 <表达式>
\end{verbatim}

玩家类型;卡牌数组类型

本语句可以直接弃置某一名玩家的相应卡牌。 \\

额外参数:

\begin{itemize}
  \item \verb|'来源'|: 玩家类型,本次弃牌的来源。比如A拆掉B的一张牌,那么语句是B弃置牌xxx,而来源是A。
  \item \verb|'技能名'|: 字符串类型,与本次弃牌相关的技能。默认为当前的技能。
\end{itemize}

\section{换牌}

\begin{verbatim}
  <表达式> 与 <表达式> 换牌
\end{verbatim}

玩家类型;玩家类型 \\

额外参数:

\begin{itemize}
  \item \verb|'区域'|: 要换牌的区域,可以为手牌区或装备区,默认为手牌区。
  \item \verb|'技能名'|: 字符串类型,与本次弃牌相关的技能。默认为当前的技能。
\end{itemize}

\section{给牌}

\begin{verbatim}
  <表达式> 交给 <表达式> 牌 <表达式>
\end{verbatim}

玩家类型;玩家类型;卡牌数组类型 \\

额外参数:

\begin{itemize}
  \item \verb|'公开'|: 布尔类型,本次给牌是否正面朝上。默认为不公开。
  \item \verb|'技能名'|: 字符串类型,与本次弃牌相关的技能。默认为当前的技能。
\end{itemize}

\section{拼点}

\begin{verbatim}
  <表达式> 与 <表达式> 拼点
\end{verbatim}

玩家类型;玩家类型 \\

额外参数:

\begin{itemize}
  \item \verb|'技能名'|: 字符串类型,与本次弃牌相关的技能。默认为当前的技能。
\end{itemize}

返回:拼点信息

参见“类型可以获取的属性”一章以详细了解如何处理拼点的结果。
