\chapter{创建武将}

\section{创建武将}

创建武将的格式如下: \\

\begin{verbatim}
 # <势力> <字符串> <标识符> <数字> [<性别>] <字符串数组>
\end{verbatim}

这一行可以创建一名武将。结合上一章的经验,我们可以知道,创建武将时需要先用“\#”开头,然后输入武将的势力、称号、名字、体力上限,最后还输入了一对方括号。

首先是势力。势力只能从\emph{魏蜀吴群神}这五个字中进行选择,并且不需要加引号。然后是武将的称号,在上面的格式里面是显示为“<字符串>”,其实就是用一对双引号包括起来的一段文本。接下来是武将的名称,是一个<标识符>,标识符是用单引号包括的文本。再接下来是武将的体力值,这里填一个数字。

再然后就是前面没有填过的几项了,武将的性别和内部id。这两项可以不填。这里顺便提一下描述格式时使用的描述方法:

\begin{itemize}
 \item \textbf{被尖括号(<>)包括的部分}:表示一个概念而不是那个文本自身。
 \item \textbf{被方括号([])包括的部分}:表示这个部分是可有可无的。
 \item \textbf{被花括号(\{\})包括的部分}:表示这个部分可以不填,也可以填许多个。
 \item 其他的就是原文照抄了。
\end{itemize}

回到创建武将的格式上,下面是“性别”这一栏。前面我们创建猪八戒的时候没有填性别,可以看出程序默认为男性武将了。性别有三个可选的选项:\emph{男性、女性、中性}。(中性这一项真的有必要存在吗= =)

最后那个方括号表示的就是武将的技能了。现在我们还没有开始写技能,所以就让它留空吧。\\

说了这么一堆,似乎给人很复杂的印象了,其实这个还是很简单的。下面给出一些例子:

\begin{verbatim}
 # 神 "----" '观世音' 3 女性 []
 # 吴 "整军经武" '谋徐盛' 4 男性 []
\end{verbatim}

上面的例子创建了两名新武将。\\

\framebox{
  \begin{minipage}{0.8\linewidth}
   我们可以注意到,创建武将的时候没有指定武将所在的拓展包。但武将属于哪个拓展包应该是很明显的,请自己去摸索吧。
  \end{minipage}
}

\section{制作武将的音画素材}

首先一个武将一共有4个素材:卡图、全身图、头图、阵亡语音。

(摸鱼了,请参考示例素材罢)
