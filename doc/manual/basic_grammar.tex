\chapter{基本语法}

这一章节介绍fkparse语言的基本语法。

\section{注释}

fkparse支持用下面的格式注释:

\begin{verbatim}
 注:
\end{verbatim}

在“注:”后面直到这一行结束的所有内容都会被视为注释,也就是被编译器忽略掉。(注意一下,这里的冒号是全角的冒号)

fkparse还支持C++和Lua风格的注释。也就是说“//”和“--”后直到行末尾的内容也都被视为注释。\\

此外,还有几个词语也会被编译器忽略:\emph{然后、立即}、中文的逗号和句号、Tab、空格、换行。毕竟“然后”和“立即”两个词语确实没啥意义,除了便于阅读之外。这意味着你可以随意滥用这两个词,反正会被忽略掉(笑

\section{标识符}

标识符用来标识变量、函数(Todo)。一个标识符是一串用一对单引号括起来的文本,它可以包含任意字符。比如\verb|'zy','犯大吴疆土者'|都是合法的标识符。

顺便一提,\verb|'你'、'X'|这两个标识符由于使用很频繁,在使用它们的时候无需打单引号。

标识符需要通过\textit{赋值语句}来定义。有关语句相关的东西我们后面再说。当然fkparse也内置了许多跟游戏概念相关的标识符供用户使用,具体见参考手册。

\section{字符串}

字符串是一串用一对双引号括起来的文本。

\section{数据类型}

fkparse定义了以下几种数据类型:

\begin{itemize}
 \item 无类型
 \item 数字类型
 \item 布尔类型(真、假)
 \item 字符串类型
 \item 玩家类型
 \item 卡牌类型
 \item 数组
\end{itemize}

这些东西无需过多操心,看一下就行了。

\section{代码块}

代码块仅仅只是一大堆语句罢了。

下一章来讨论fkparse所支持的语句。


