\chapter{创建技能}

\section{先来创建一个简单的技能}

只能制作白板武将的话,这个工具的意义就不大了。制作各种各样的技能才是fkparse的最主要的功能。

由于技能和武将是独立的,但需要依附拓展包而存在,所以fkparse中规定技能的定义必须放在某个拓展包定义的后面。

现在打开study.txt,在第一个拓展包的后面起一个新行,然后输入以下内容:

\begin{verbatim}
$ '吃苦' "锁定技,你受到伤害后,摸一张牌。" 锁定技
  触发技
  时机:受到伤害后
  效果:你摸1张牌
以上
\end{verbatim}

这样就创建了一个名为“吃苦”的技能。但是光有技能还不行,还得有武将拥有这个技能,不然意义就不大啦。

接下来去我们创建猪八戒的那一行,修改一下,让他获得新技能:

\begin{verbatim}
 # 神 "净坛使者" '猪八戒' 24 ["吃苦"]
\end{verbatim}

现在的study.txt应该像这样,保存一下然后生成lua代码吧。

\begin{verbatim}
拓展包 '学习包1'

$ '吃苦' "锁定技,你受到伤害后,摸一张牌。" 锁定技
  触发技
  时机:受到伤害后
  效果:你摸1张牌
以上

# 神 "净坛使者" '猪八戒' 24 ["吃苦"]
# 神 "----" '观世音' 3 女性 []
# 吴 "整军经武" '谋徐盛' 4 男性 []
拓展包 '学习包2'
\end{verbatim}

\section{创建技能的一般格式}

创建技能的格式如下:

\begin{verbatim}
 $ <标识符> <字符串> [<技能频率>] <技能主体内容> 以上
\end{verbatim}

其中,<标识符>是技能名字,<字符串>是技能描述。技能频率可填可不填。

所谓技能的发动频率,就是锁定技、限定技之类的标签而已。技能频率有以下五种取值:\emph{普通技、锁定技、默认技、觉醒技、限定技}。普通技就是普通的技能,而所谓默认技就是在询问发动时直接默认发动(比如枭姬、奸雄之类的技能)。

剩下最重要的部分就是技能的主体内容了。正如前面所示,单纯写个技能描述是不能达到制作新技能的目的的。我们需要按照一定的语法规范去编写技能的主体内容,才能让fkparse生成相应的技能代码。

\textbf{技能主体内容}又分为这几大类型:触发型技能、主动型技能、状态型技能。

\section{触发型技能}

触发型技能的创建格式如下:

\begin{verbatim}
 触发技 时机:<时机> [条件:<代码块>] 效果:<代码块>
  {时机:<时机> [条件:<代码块>] 效果:<代码块>}
\end{verbatim}

首先以“触发技”三个字开头,之后跟上许多时机和效果,这就是触发技的创建方法。fkparse支持的时机有许多种,在用户指南部分我们不一一罗列,后面会列出所有时机的。现在只需要知道示例里面的时机即可。

然后是技能的触发条件。这个可以不填,它的默认处理方法是判断当前时机的角色是不是玩家,以及玩家是否拥有本技能。我们直接来看技能的效果吧。效果后面跟的是<代码块>,也就是说我们要开始“编程”了,用fkparse和我们自己能看懂的方式。代码块并不难看懂,比如我们前面写过的一句代码“你摸1张牌”,这个一看字面就知道是什么意思啦。\\

接下来的几章会结合实例,详细介绍fkparse能接受的代码格式。
